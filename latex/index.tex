Physical units, time, reference frames, environment modeling.

\href{https://travis-ci.com/open-space-collective/library-physics}{\tt } \href{https://codecov.io/gh/open-space-collective/library-physics}{\tt } \href{https://open-space-collective.github.io/library-physics}{\tt } \href{https://badge.fury.io/gh/open-space-collective%2Flibrary-physics}{\tt } \href{https://badge.fury.io/py/LibraryPhysicsPy}{\tt } \href{https://opensource.org/licenses/Apache-2.0}{\tt }

\subsection*{Warning}

Library {\bfseries name} is yet to be defined.

Please check the following projects\+:


\begin{DoxyItemize}
\item \href{https://github.com/orgs/open-space-collective/projects/1}{\tt Naming Project}
\end{DoxyItemize}

{\itshape ⚠ This library is still under heavy development. Do not use!}

\subsection*{Structure}

The {\bfseries Physics} library exhibits the following structure\+:


\begin{DoxyCode}
├── Units
│   ├── Length
│   ├── Mass
│   ├── Time
│   ├── Temperature
│   ├── Electric Current
│   ├── Luminous Intensity
│   └── Derived
│       ├── Angle
│       ├── Solid Angle
│       ├── Frequency
│       ├── Force
│       ├── Pressure
│       ├── Area
│       ├── Volume
│       └── Information
├── Time
│   ├── \hyperlink{namespacelibrary_1_1physics_1_1time_a09d2bc9fbc7b0b5f92e1419bd655e6bb}{Scale} (UTC, TT, TAI, UT1, TCG, TCB, TDB, GMST, GPST, GST, GLST, BDT, QZSST, IRNSST)
│   ├── Instant
│   ├── Duration
│   ├── Interval
│   ├── Date
│   ├── Time
│   └── DateTime
├── Coordinate
│   ├── Transform
│   └── Frame (ECI, ECEF, NED, LVLHGD, LVLHGDGT, ...)
├── Geographic
│   ├── Position
│   ├── Area
│   ├── Volume
│   ├── Coordinate Reference System (CRS)
│   └── Universal Transverse Mercator (UTM)
└── Environment
    ├── Constants
    ├── Object
    │   └── Celestial
    ├── Ephemerides
    │   ├── Analytical
    │   ├── Tabulated
    │   ├── SOFA
    │   └── SPICE (JPL)
    ├── Gravity
    │   ├── Barycentric
    │   ├── Earth Gravitational Model 1996 (EGM96)
    │   └── Earth Gravitational Model 2008 (EGM2008)
    ├── Atmospheric
    │   ├── Exponential
    │   ├── USSA1976
    │   ├── Jacchia Roberts
    │   └── NRLMSISE00
    ├── Magnetic
    │   ├── Dipole
    │   ├── World Magnetic Model 2010 (WMM2010)
    │   ├── World Magnetic Model 2015 (WMM2015)
    │   ├── Enhanced Magnetic Model 2010 (EMM2010)
    │   ├── Enhanced Magnetic Model 2015 (EMM2015)
    │   ├── International Geomagnetic Reference Field 11 (IGRF11)
    │   └── International Geomagnetic Reference Field 12 (IGRF12)
    ├── Radiation
    │   └── Sun Static
    └── Stars
        └── Hipparcos
\end{DoxyCode}


\subsection*{Documentation}

The documentation can be found here\+:


\begin{DoxyItemize}
\item \href{https://open-space-collective.github.io/library-physics}{\tt C++}
\item \href{./bindings/python/docs}{\tt Python}
\end{DoxyItemize}

\subsection*{Tutorials}

Various tutorials are available here\+:


\begin{DoxyItemize}
\item \href{./tutorials/cpp}{\tt C++}
\item \href{./tutorials/python}{\tt Python}
\end{DoxyItemize}

\subsection*{Setup}

\subsubsection*{Development}

Using \href{https://www.docker.com}{\tt Docker} is recommended, as the development tools and dependencies setup is described in the provided \href{./tools/development/docker/Dockerfile}{\tt Dockerfile}.

Instructions to install Docker can be found \href{https://docs.docker.com/install/}{\tt here}.

Start the development environment\+:


\begin{DoxyCode}
./tools/development/start.sh
\end{DoxyCode}


This will also build the {\ttfamily openspacecollective/library-\/physics\+:latest} Docker image, if not already.

If installing Docker is not an option, please manually install the development tools (G\+CC, C\+Make) and the dependencies. The procedure should be similar to the one described in the \href{./tools/development/docker/Dockerfile}{\tt Dockerfile}.

\subsubsection*{Build}

From the development environment\+:


\begin{DoxyCode}
./build.sh
\end{DoxyCode}


Manually\+:


\begin{DoxyCode}
mkdir -p build
cd build
cmake ..
make
\end{DoxyCode}


\subsubsection*{Test}

From the development environment\+:


\begin{DoxyCode}
./test.sh
\end{DoxyCode}


Manually\+:


\begin{DoxyCode}
./bin/library-physics.test
\end{DoxyCode}


\subsection*{Dependencies}

The {\bfseries Physics} library internally uses the following dependencies\+:

\tabulinesep=1mm
\begin{longtabu} spread 0pt [c]{*{4}{|X[-1]}|}
\hline
\rowcolor{\tableheadbgcolor}\textbf{ Name }&\textbf{ Version }&\textbf{ License }&\textbf{ Link  }\\\cline{1-4}
\endfirsthead
\hline
\endfoot
\hline
\rowcolor{\tableheadbgcolor}\textbf{ Name }&\textbf{ Version }&\textbf{ License }&\textbf{ Link  }\\\cline{1-4}
\endhead
Boost &1.\+67.\+0 &Boost Software License &\href{https://www.boost.org}{\tt boost.\+org} \\\cline{1-4}
Eigen &3.\+3.\+4 &M\+P\+L2 &\href{http://eigen.tuxfamily.org/index.php}{\tt eigen.\+tuxfamily.\+org} \\\cline{1-4}
I\+AU S\+O\+FA &2018-\/01-\/30 &\href{http://www.iausofa.org/tandc.html}{\tt S\+O\+FA Software License} &\href{http://www.iausofa.org}{\tt www.\+iausofa.\+org} \\\cline{1-4}
S\+P\+I\+CE Toolkit &N0066 &\href{https://naif.jpl.nasa.gov/naif/rules.html}{\tt N\+A\+IF} &\href{https://naif.jpl.nasa.gov/naif/toolkit.html}{\tt naif.\+jpl.\+nasa.\+gov/naif/toolkit.html} \\\cline{1-4}
Geographic\+Lib &1.\+49 &M\+IT &\href{https://geographiclib.sourceforge.io}{\tt geographiclib.\+sourceforge.\+io} \\\cline{1-4}
Core &master &Apache License 2.\+0 &\href{https://github.com/open-space-collective/library-core}{\tt github.\+com/open-\/space-\/collective/library-\/core} \\\cline{1-4}
I/O &master &Apache License 2.\+0 &\href{https://github.com/open-space-collective/library-io}{\tt github.\+com/open-\/space-\/collective/library-\/io} \\\cline{1-4}
Mathematics &master &Apache License 2.\+0 &\href{https://github.com/open-space-collective/library-mathematics}{\tt github.\+com/open-\/space-\/collective/library-\/mathematics} \\\cline{1-4}
\end{longtabu}
\subsection*{Contribution}

Contributions are more than welcome!

Please read our \hyperlink{_c_o_n_t_r_i_b_u_t_i_n_g_8md}{contributing guide} to learn about our development process, how to propose fixes and improvements, and how to build and test the code.

\subsection*{Special Thanks}

{\itshape To be completed...}

\subsection*{License}

Apache License 2.\+0 